\chapter{Markov chains and random walks}

We now turn to algorithms that can be analyzed using random walks on graphs. We
will use the theory of Markov chains to analyze these random walks. We will
start with the following algorithmic question.

Consider a boolean formula $\phi$ in 2-CNF form. I.e.\ $\phi$ can be written as
$C_1 \wedge C_2 \wedge \cdots \wedge C_m$ where each $C_i$ is a disjuction on
two literals, say $\ell_1 \vee \ell_2$. A literal is a boolean variable or its
negation. We will describle a simple algorithm and then explain how this
algorithm can be analyzed using a random walk on a suitable graph.

\section{2-SAT}

Let $\phi = C_1 \wedge C_2 \wedge \cdots \wedge C_m$ where $C_i$ is a
disjunction of at most two literals. This is known as the 2-SAT problem. This
has a simple polynomial-time deterministic algorithm, where we can model the
formula using a directed graph whose vertices are the variables and their
negations. For each clause $C = \ell \vee \ell'$, we put an edge from
$\neg \ell \to \ell'$ and from $\neg \ell' \to \ell$ since we can can represent
a disjunction equivalently using implication. Now, it is easy to verify that the
formula $\phi$ is satisfiably iff no strongly connected component of this
digraph contains both a variable and its negation. Thus, there is a linear-time
deterministic algorithm for 2-SAT.

We will, on the other hand, see a simple randomized algorithm for this
problem. The analysis of the algorithm will help us to generalize this to an
algorithm for 3-SAT that, even though exponential, will be faster than the
brute-force algorithm.

The 2-SAT algorithm is simple to describe: we start with an \emph{arbitrary}
assignment $\bar{a}$, say the all-zero assignment, and check if it satisfies
$\phi$. If it does, then we stop. Else if $\phi(\bar{a}) = 0$, then there is
some $C_i$ such that $C_i(\bar{a}) = 0$. Choose one of the literals in this
clause $C_i$ u.a.r and flip it. \marginnote{We will fix $r$ during the
  analysis.}Keep continuing from some $r$ steps until you find a satisfying
assignment. If you haven't found one until then, then stop and return that
$\phi$ is not satisfiable.

Notice that this algorithm has the potential to loop infintely by going over the
same set of assignments, even though the formula is satisfiable. Consider this
example:
\begin{align*}
  \phi = (x_1 \vee \bar{x}_2) \wedge (x_2 \vee \bar{x}_3) \wedge (x_3 \vee x_1).
\end{align*}

Notice that $\bar{a} = 000$ is not a satisfying assignment since $C_3(000) =
0$. So, suppose we flip $x_3$ to $1$, then we have a new assignment
$\bar{a'} = 001$. But $C_2(\bar{a'}) = 0$. There is a non-zero probability of
flipping $x_3$ again, and this can lead to a sequence of flips that that cycles
between these assignments, never converging to a satisfying assignment. Instead
of arguing that the algorithm will always reach a satisfying assignment, if one
exists, we will instead argue that the algorithm is more likely to move towards
a satisfying assignment that away from it.

%%% Local Variables: 
%%% mode: latex
%%% TeX-master: "notes"
%%% End:
